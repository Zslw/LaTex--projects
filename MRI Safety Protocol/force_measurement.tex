% DRAFT MODE: Uncomment this for a more readable single-column draft
\documentclass[aps,prl,preprint,superscriptaddress]{revtex4-2}

% FINAL JOURNAL FORMAT: Uncomment this and comment the above line for submission format
% \documentclass[aps,prl,twocolumn,superscriptaddress]{revtex4-2}

\usepackage{graphicx}
\usepackage{amsmath}
\usepackage{siunitx}
%\usepackage{hyperref}

% Optional: improve appearance in draft mode
\usepackage[margin=1in]{geometry}    % Adjusts margins to make draft more readable
\linespread{1.25}                    % Slightly more generous line spacing for drafts

% Optional: mark the document as a draft
 \usepackage{draftwatermark}
 \SetWatermarkText{DRAFT}
 \SetWatermarkScale{1.5}

\begin{document}
	
	\title{Measurement of Magnetically Induced Forces on Lightweight Medical Devices in the MRI Environment}
	
	\author{Daniel Juarez Luna}
	\affiliation{Creighton University}
	\email{danieljureluna@creighton.edu}



\begin{abstract}
Scope: 	This protocol provides a standardized methodology to assess MRI-induced translational forces on retained or temporary epicardial pacing leads using a pendulum setup in both laboratory and MRI environments. It supports preclinical lead evaluation and translation to patient-specific MRI clearance, enabling communication between physicists and radiologists.
	
Objective: 

Methods: 

Results: 

Conclusion: 
\end{abstract}


%keywords: ASTM F2052, MRI safety, translational force, lightweight devices, MR Conditional, magnetic resonance.

\maketitle
\section{Introduction}
% ============================
% Section: Introduction
% ============================

%\subsection{The Role of Imaging in Radiation Therapy}

During all stages of radiotherapy, getting the sense of where health professionals focus their work and the methods used to shape or conform the target tissue is of most vital importance. This information is provided by different imaging devices and techniques molded to each different case. During planning, images help visualize target lesions and clarify which pathway and devices are best to use. This information serves as a guide to determine patient and device position. During treatment, it provides insight into treatment outcomes and potential complications \cite{decazes2021}. Finally, during surveillance or control, it gives the comparison point to check accuracy and success while pointing to the right direction to follow.

The preferred imaging modality for all types of cancer diagnosis is Computed Tomography (CT), this is due to it high-resolution anatomical detail and the information provided to map tissue electron density \cite{decazes2021,yan2024}. However, like every imaging modality, CT has its limitations, emphazising the need for complementary imaging techniques, especially for precise tumor targeting in complex cases like liver cancer \cite{yan2024}.

%\subsection{Hybrid Imaging for Enhanced Accuracy}
For these reason, no single imaging modality is entirely sufficient for an absolute precise role in radiation therapy due to intrinsic trade-offs \cite{decazes2021}.Setting CT aside, another good alternative is Magnetic Resonance Imaging (MRI), although valuable for soft tissue delineation, it lacks electron density information, making them less suitable as a standalone modalities in radiotherapy \cite{decazes2021}. Another alternative is Positron Emission Tomography (PET), which provides a new lens since it contributes with unique metabolic information that complements anatomical imaging and aids in tumor delineation \cite{decazes2021}.

Exploring new techniques, technologies and pathways of using existing devices are carefully studied. Hybrid techniques are of most interest such as PET/CT and PET/MR. These modalities aim to overcome limitations by combining the strengths of each modality.

The objective of this paper is to compare PET/CT and PET/MR in liver cancer radiotherapy, focusing on their applications in tumor delineation and dosimetric planning, and patient monitoring. An evaluation of their technical capabilities, clinical integration, and implications for adaptive radiotherapy workflows will be inlcuded. The aim is to explore how these hybrid modalities can enhance precision in liver cancer management. 

Table \ref{tab:modality_comparison} provides a comparative overview of the relevant used imaging modalities in radiotherapy, it includes technical complexity and workflow integration that may be overlooked. Each modality has specific roles, because of it no single technique is entirely sufficient for precise tumor targeting. Following the table an explanation of each modality and why hybrid approaches are needed is explained.

\end{multicols}
%merge Decazes and Yan
%Decazes TABLE 1 | Comparison of PET/MRI and other conventional image-based modalities in radiotherapy
%Yans TABLE 1 | Summary of advantages and disadvantages of CT, MRI, and PET separately and combined in PET/CT/MRI
%Yans TABLE 3 | Comparison of different imaging methods (star ranking)

\begin{landscape}
\begin{table}[H]
	\centering
	\renewcommand{\arraystretch}{1.5} % Adjust the multiplier (e.g., 1.5 for 50% more spacing)
	\setlength{\tabcolsep}{6pt} % Optional: Adjust column padding if needed
	\begin{tabular}{|l|>{\raggedright\arraybackslash}p{3.5cm}|>{\raggedright\arraybackslash}p{2.3cm}|>{\raggedright\arraybackslash}p{5.5cm}|>{\raggedright\arraybackslash}p{4.5cm}|>{\raggedright\arraybackslash}p{4.5cm}|}
		\hline
		\textbf{Modality} & \textbf{Diagnostic Information} & \textbf{Soft Tissue Resolution} & \textbf{Radiation Dose} & \textbf{Technical Complexity} & \textbf{Workflow Integration} \\ \hline
		CT & Anatomy and tissue electron density & Moderate (1 mm spatial resolution) & Ionizing radiation (higher dose) & Widely available; simple infrastructure requirements & Integrates easily into standard radiotherapy workflows; rapid acquisition \\ \hline
		MRI & Anatomy and function & Excellent (soft tissue contrast) & None (non-ionizing) & Requires trained personnel; specialized infrastructure & Long scan times; challenges with integration into dose planning workflows \\ \hline
		PET & Metabolic and functional imaging & Poor (blurred edges, partial volume effects) & Ionizing radiation (tracer-based) & Limited isotope availability; moderate complexity   & Requires co-registration with CT or MRI; \\ \hline
		PET/CT & Combines anatomical and functional imaging; provides BTV and GTV delineation & Moderate & Higher due to combined CT dose & Standardized; relatively simple calibration & Commonly used in clinical settings \\ \hline
		PET/MRI & Combines superior soft tissue contrast with metabolic imaging; pseudo-CT for dose calculation & Excellent & Reduced radiation compared to PET/CT & High complexity; limited infrastructure and operator expertise & Workflow integration less standardized compared to PET/CT \\ \hline
	\end{tabular}
	\caption{Comparison of imaging modalities (CT, MRI, PET, PET/CT, and PET/MRI) for radiotherapy applications. Adapted from Decazes et al. \cite{decazes2021}, Yan et al. \cite{yan2024}, and Knešaurek et al. \cite{knesaurek2018}.}
	\label{tab:modality_comparison}
\end{table}
\end{landscape}

\begin{multicols}{2}

First, CT is the standard modality used in radiotherapy due to its high anatomical resolution and integration into treatment workflows. However, it relies on ionizing radiation and it has moderate soft tissue resolution that limits its standalone use in complex cases, such as tumors adjacent to sensitive soft tissues. 

Next, MRI provides a solution to CT soft tissue resolution and adds functional imaging without ionizing radiation, making it ideal for delineating tumors in challenging areas.  Nevertheless, its known for long scan times and the difficulty to integrate electron density information of the tissue on the treatment plan. 

Then, PET offers a unique insight on tumor activity, that is functional and metabolic imaging. Still, its limitations are quite considerable, mainly poor spatial resolution which makes this modality rely on co-registration with anatomical modalities like CT or MRI. 

Following that, PET/CT solves these limitations by combining the anatomical clarity of CT with the functional insights of PET. Yet, the combined radiation dose remains a concern. 

Finally, PET/MRI merges superior soft tissue delineation with metabolic imaging. This modality reduces radiation exposure compared to PET/CT but faces significant challenges in technical complexity, cost, and workflow integration.

Imaging modalieties are complementary in nature because there are trade offs and risks on each of them. While CT and MRI is great in anatomical and soft tissue imaging, respectively, and PET provides metabolic insights, their integration through hybrid systems like PET/CT and PET/MRI is essential for achieving optimal precision in radiotherapy.\cite{decazes2021, yan2024}. Table \ref{tab:modality_comparison} convey the importance of advancement in hybird techniques to overcome the limitations of individual modalities and confidentely approach any complex clincal scenarios with either IMRT or VMAT as treatment.


% Add your text here

\section{Methods}
% Add your text here
%\section{I. Laboratory Protocol (Physicist Focus)}
\subsection*{Materials and Setup}
\begin{itemize}
	\item \textbf{PVC Base:} Rectangular prism with leveling screws.
	\item \textbf{Pendulum Arm:} 1.5 m vertical tube, light string suspension.
	\item \textbf{Lead Mount:} PVC frame replacing magnetic rod, with fixed hangers.
	\item \textbf{Ruler:} Positioned behind pendulum to measure z-displacement.
	\item \textbf{Solenoid:} Coil capable of up to 5 A current. Represents MRI B$_0$ gradient.
	\item \textbf{Optional Guide Rails:} PVC slots to limit pendulum rotation and allow angle testing.
\end{itemize}

\subsection*{Procedure}
\begin{enumerate}
	\item Level base with screws; confirm vertical alignment of the tube.
	\item Attach the lead to the frame and ensure central positioning.
	\item At each test position (distance from solenoid), record resting position.
	\item Energize solenoid and record max z-displacement.
	\item Compute angle: $\theta = \arcsin(\Delta z / L)$
	\item Compute force: $F = mg \tan(\theta)$
	\item Repeat 3--5 times per position; calculate mean, standard deviation.
\end{enumerate}

\subsection*{Uncertainty Notes}
\begin{itemize}
	\item Waiting for swing stabilization increases precision, but slows throughput.
	\item Trade-off documented: rapid swings accepted, wider uncertainty bounds.
	\item High current causes sudden attraction $\rightarrow$ mitigate by spacing solenoid.
\end{itemize}

\section{II. Clinical Translation}
\begin{itemize}
	\item In MRI, B$_0$ is always on; increasing solenoid current models proximity to bore.
	\item Couch marks every 10--15 cm simulate test spacing.
	\item Rotational setup mirrors patient-specific orientations (axial, sagittal, oblique).
\end{itemize}

\section{III. Radiologist Collaboration}
\subsection*{MRI Clearance Flowchart (ASCII-safe)}
\begin{verbatim}
	[Patient has epicardial lead]
	|
	[Is lead model known?]
	|
	Yes   No
	|     |
	[Check]   [Refer to Physicist]
	|
	[Force < 0.01 N?]
	|
	Yes   No
	|     |
	[MRI OK] [MRI Contraindicated]
\end{verbatim}

\subsection*{Workflow Proposal}
\begin{itemize}
	\item Maintain internal registry with lead ID, orientation, max force, test conditions.
	\item Share summaries as PDFs; link to PACS or EPIC where feasible.
	\item Annual Radiology--Physics review to update protocols and registry.
\end{itemize}

\section*{Appendices (To Be Added)}
\begin{itemize}
	\item \textbf{Appendix A: Mechanical Setup} -- diagrams, photos, 3D model links.
	\item \textbf{Appendix B: Calibration Logs} -- pendulum swing logs, trigonometry, force estimates.
	\item \textbf{Appendix C: Clinical Registry Template} -- lead name, test ID, orientation, force, MRI recommendation.
\end{itemize}
%\section{Materials and Methods}

%\subsection{Purpose and Overview}

The objective of this study was to evaluate MRI-induced translational forces acting on post-surgical epicardial pacing leads using in-lab simulations based on ASTM F2052-15 standards. A secondary goal was to develop and validate apparatuses that could later be used in clinical settings to replicate these measurements under MRI field conditions. Three distinct experimental setups were involved in this study: (1) a spatial field mapping platform to characterize $B_0$ and $\nabla B_0$, (2) a pendulum-based deflection apparatus to assess translational force through angular displacement, and (3) a torque measurement system for rotational force assessments, the latter of which remains under evaluation.

\subsection{Magnetic Field and Gradient Mapping Platform}

Magnetic field strength ($B_0$) and gradient ($\nabla B_0$) measurements were performed using a commercial Hall-effect Gaussmeter (Model GM2, AlphaLab Inc., Salt Lake City, UT).  The GM2 operates on the Hall effect principle, wherein a voltage is generated across a probe when exposed to a perpendicular magnetic field. This voltage is proportional to the local magnetic flux density, enabling precise spot measurements of field magnitude. The Gaussmeter used has an accuracy of 1\% of the DC reading in the 16$^\circ$C to 29$^\circ$C range.

To ensure spatial accuracy and reproducibility, a fixture made from rigid PVC was constructed. The platform comprises a 300 mm square base and a 280 mm diameter vertical circular plate, perforated with a 5x5 grid of probe-access holes spaced at 20 mm intervals. Each access point was numerically indexed to facilitate 2D field mapping in the X-Y plane.

In clinical use, the entire fixture is intended to be shifted axially along the Z-direction of a horizontal bore MRI scanner using controlled couch increments (e.g., 10 cm). This translation allows for volumetric sampling of $B_0$ over several planes. The spatial magnetic field gradient$\nabla B_0$ can then be estimated numerically using finite difference calculations between adjacent measurement points. This method has previously been applied to MRI mapping in the work by Ferreira \cite{ferreira2017}.

\subsection{Translational Force Measurement via Angular Deflection}

\subsubsection{Experimental Setup}

\subsubsection{Laboratory Simulation with Solenoid Coil}

To assess MRI-induced translational forces in a laboratory environment, we developed a pendulum-style apparatus replicating the geometry outlined in ASTM F2052-15~\cite{astm2052}. The vertical post includes a screw clamp mechanism that enables height adjustment; however, a default suspension height of 1 meter was used for consistency across trials.

The experimental frame was constructed entirely from \textbf{PVC} components (Figure~\ref{fig:schematic}), including a vertical post of 1.5 meters and a rigid base for stabilization. The sample—a \textbf{solid ferromagnetic cylinder} (length = 6 mm, diameter = 3 mm, mass = 0.3648 g, density = 8.6 g/cm$^3$)—was suspended by a \textbf{955 mm-long sewing thread}, tied via a simple knot at its top. For clinical applications, a \textbf{0.25 mm nylon monofilament} is recommended for improved reproducibility and safety. Additionally a frame for lead configuration must be added, which design is discussed briefly in the next section. The bottom of the sample was free to swing, forming a pendulum under gravity. A horizontal ruler, fixed along the Z-axis, served as the visual reference to quantify lateral displacements as shown in Figure~\ref{fig:schematic} b .


%add the 3d model and actual thing photo

\subsubsection*{Lead Holder Frame Design}

To accommodate epicardial lead positioning while maintaining MRI compatibility, we propose the use of a lightweight, non-magnetic frame fabricated from Polyvinyl Chloride (PVC) and equipped with nylon screws used to secure the epicardial lead at discrete positions. The suggested frame dimensions are 70 mm wide by 100 mm tall, with a thickness of 3 mm. A series of 3 mm diameter holes for screws placed along the vertical centerline at 10 mm intervals should allow reproducible placement of leads at known lengths (6–13 cm). As shown in figure [smth]

The frame needs to be suspended such that the lead axis was aligned approximately parallel to the magnetic field gradient (Z-axis). While the frame's lightweight design aimed to minimize its contribution to the total magnetic force, its mass may not be explicitly subtracted from the translational force calculations.

%add the custom design

\subsubsection*{Magnetic Field Generation}

Magnetic fields were generated using a \textbf{custom-built air-core solenoid}, approximately 15 cm in length and 10 cm in outer diameter. The solenoid was densely wound with \textbf{copper wire ($\sim$1 mm diameter)}, forming an inner core of approximately 5 cm. A \textbf{Pasco Scientific SF-9584 DC power supply} was used to deliver currents in the range of \textbf{0--4 A}, corresponding to magnetic field strengths from \textbf{1 to 18.9 Gauss}. The system operated in a continuous mode; no rest periods were applied between trials due to the time required to stabilize the pendulum after each adjustment. as shown in the experimental setup.

%add the figure from pptx

Magnetic field measurements were performed using an \textbf{AlphaLab Inc. Gaussmeter}. To characterize local field gradients, measurements were taken at three positions: at the pendulum's equilibrium position, 5 mm below, and 5 mm above, yielding values of $B_{-5}$, $B_0$, $B_{+5}$, respectively. Each position was sampled \textbf{nine times per current level}, and gradients were computed using the finite difference method:

\begin{equation}
	\frac{dB}{dz} \approx \frac{B_{+5} - B_{-5}}{10 mm}
\end{equation}

\subsection*{Measurement Protocol}

Once the pendulum reached equilibrium under each magnetic condition, the horizontal displacement $x$ was recorded visually using the ruler as a reference. With known suspension length $L$, the \textbf{deflection angle $\alpha$} was calculated as:

\begin{equation}
	\alpha = \tan^{-1}\left(\frac{x}{L}\right)
\end{equation}

Following ASTM F2052-15, the ratio of magnetic to gravitational force is:

\begin{equation}
	\tan \alpha = \frac{F_m}{mg}
\end{equation}

where $m$ is the mass of the test object and $g$ is gravitational acceleration. The translational force $F_m$ is then:

\begin{equation}
	F_m = mg \tan \alpha
\end{equation}

This relation also enabled the \textbf{estimation of volumetric magnetic susceptibility} $\chi$ under the assumption of a linear force model:

\begin{equation}
	F_m = \frac{\chi V}{\mu_0} B_0 \frac{dB_0}{dz}
	\quad \Rightarrow \quad
	\chi = \frac{\mu_0 F_m}{V B_0 \frac{dB_0}{dz}}
\end{equation}

where $\mu_0$ is the permeability of free space and $V$ is the object volume. Values of $\chi$ were inferred from the slope of $\tan \alpha$ versus $B_0 \frac{dB_0}{dz}$, based on the Newtonian framework.

\subsection*{Data Collection and Analysis}

All measurements were logged in \textbf{Microsoft Excel}, and photos/videos were casually captured via smartphone to document behavior. A minimum of \textbf{three trials} were performed for each current level; however, only the final (most consistent) dataset was retained for analysis. Data were plotted to evaluate $\tan \alpha$ vs. field gradient force. Propagated uncertainties in $\alpha$ and $dB_0/dz$ were computed using standard error propagation formulas. Estimated measurement errors were:

\begin{itemize}
	\item Gaussmeter resolution: $\pm$0.001 G  
	\item Ruler displacement: $\pm$1 mm  
\end{itemize}

A summary schematic of the apparatus is included in \textbf{Figure~\ref{fig:schematic}}, with additional engineering drawings available in \textbf{Appendix A}.


\subsection{Torque Measurement Apparatus (In Progress)}

A torque measurement system based on ASTM F2213 is under development. While a prototype was fabricated under the supervision of Dr. Nichols and prior students, detailed calibration, operational workflow, and data acquisition methods are still being finalized. The system is expected to allow both qualitative torque threshold assessments and quantitative torque calculation by assessing rotational displacement under a known static field.



\section{Results}
% Add your text here

\section{Discussion}
% Add your text here


EVALUATION PLAN
Device Fabrication and Protocol Development. After we have fabricated the devices described in Aims 1 and 2 and
established their utility, we will prepare publications providing appropriate schematics and materials as well as protocols
for their effective use. This will also be disseminated through presentations at local, regional, and national conferences.

Data Analysis. The successful completion of the project requires establishing the measurement accuracy for the static B0
field, the field gradient, translational forces, and torques (by two techniques). For each of 10 Z-axis locations, approximately
60 measurements of B0 will be made and repeated on at least three separate occasions to verify consistency. Each of the
translational force and torque measurements will be repeated approximately 3-5 times at each location within the MRI
scanner tested (up to 10 axial measurement sites with 5 positions within the plane (left, middle, right, top, bottom). The
mean and the standard error will be reported and results for each epicardial lead will be compared to the established safety
standards.

Expected Outcomes, Benchmarks for Success, and Projecting Future Directions. By completing Aims 1 and 2, we will
develop the apparatus and methodology to 1) characterize the spatial distribution of the magnetic field in a clinical MRI
scanner, 2) measure the maximum translational forces and torques on a representative subset of temporary, post-surgical
epicardial pacing leadings, and 3) assess the safety of the leads regarding static B0 magnetic field interactions according to
established standards. These results will be interpreted along with our prior measurements of potential hazards of RF heating
to provide a more complete assessment. The techniques and the collaboration strengthened through this study will also pave
the way for a future study of the potential for electrical stimulation by retained, post-surgical epicardial pacing leads during
MRI. We plan to publish these results within respected medical physics and radiology journals, and will disseminate the
results at national meetings, including the American Association of Physicists in Medicine (AAPM) annual meeting.
Progress made through this work will enable applications for future funding such as NIH R15 and R21 proposals through
NIGMS and NIBIB.


\section{Conclusion}
% Add your text here

\bibliographystyle{apsrev4-2}
\bibliography{references}

\end{document}
