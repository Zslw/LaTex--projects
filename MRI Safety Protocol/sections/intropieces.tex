%----------------------------------------------------------
% From Stoianovici2023
%----------------------------------------------------------


The magnetic resonance (MR) environment imposes
stringent constraints and requirements on the use of
associated medical devices [1]. 

Several internationally recognized standards relate to the safety of using devices in the
MR environment [2-4]. 

ASTM F2503-13 [2] classifies devices
as MR Safe, Conditional, and Unsafe. For the MR Safe class,
testing may be circumvented by the scientific rational that
devices exclusively made of MR Safe materials (nonmagnetic
and dielectric) are MR Safe. For other classes, however, devices
require testing. Tests include the magnetically induced force
(ASTM F2052 [5]) and torque (ASTM F2213 [6]) that are
known to be induced by the static magnetic field of MRI on
other than MR Safe devices. If the maximum magnetically
induced force and torque are less than the force and torque on
the device due to gravity respectively, it is assumed that any
risk imposed by the application of the magnetically induced
effect is no greater than any risk imposed by normal daily
activity in the Earth’s gravitational field [5, 6]. 

In specific cases,
greater magnetically induced effects may be acceptable and
would not harm patient or personnel. However, the gravity
reference gives the most conservative acceptance criterion for
an MR Conditional classification.
Under the gravity reference, the maximum magnetically
induced displacement force should be less than the device
weight (Sec 5.2 [5]), and the maximum magnetically induced
torque should be less than the product of the longest dimension
of the medical device and its weight (Sec 5.3 [6]). Tests are to
be conducted in MR systems with a horizontal bore.
For the force test the ASTM gives a single method and simple
apparatus: the test device is suspended by a string at the
entrance and on the axis of the MR bore. Trivially, the
magnetically induced deflection angle of the string should be
less than 45$\deg$.

1] D. Stoianovici, C. Kim, D. Petrisor, C. Jun, S. Lim, M. W. Ball, A. E. Ross,
K. Macura, and M. E. Allaf, "MR Safe Robot, FDA Clearance, Safety, and
Feasibility of Prostate Biopsy Clinical Trial," IEEE-ASME Transactions on
Mechatronics, vol. 22, pp. 115-126, 2017.
http://urobotics.urology.jhu.edu/pub/2017-stoianovici-tmech.pdf
[2] ASTM International, "F2503-13," in Standard Practice for Marking
Medical Devices and Other Items for Safety in the Magnetic Resonance
Environment: 100 Barr Harbor Drive, PO Box C700, West Conshohocken,
PA 19428-2959. United States, 2020. https://www.astm.org/f2503-
13.html
[3] International Organization for Standardization, "ISO TS 10974 :
Assessment of the safety of magnetic resonance imaging for patients with
an active implantable medical device," 2018.
https://www.iso.org/standard/65055.html
[4] U.S. Food \& Drug Administration, "MRI Information for Industry.
Standards Relevant to MRI Safety." https://www.fda.gov/radiation-
emitting-products/mri-magnetic-resonance-imaging/mri-information-
industry
[5] ASTM International, "F2052-21," in Standard Test Method for
Measurement of Magnetically Induced Displacement Force on Medical
Devices in the Magnetic Resonance Environment: 100 Barr Harbor Drive,
PO Box C700, West Conshohocken, PA 19428-2959. United States, 2022.
https://www.astm.org/f2052-21.html
[6] ASTM International, "F2213-17," in Standard Test Method for
Measurement of Magnetically Induced Torque on Medical Devices in the
Magnetic Resonance Environment: 100 Barr Harbor Drive, PO Box C700,
West Conshohocken, PA 19428-2959. United States, 2017.
https://www.astm.org/f2213-17.html

%----------------------------------------------------------
%From Haddix proposal, our work at creighton (previous to me)
%----------------------------------------------------------
More than 4.5 million people worldwide live with cardiac implantable electronic devices (CIEDs) such as pacemakers to
help normalize heart rhythm but delivering electrical pulses to the heart [7]. 

In addition to permanent pacemakers, epicardial
pacing leads are often placed on the heart during cardiac surgery to temporarily regulate the pacing of the heart using an
external pacing device [8]. 

While intended to be temporary, removing these leads can be problematic and for patient safety
are often cut off at the skin, leaving a relatively short length of wire within the patient [4].

 However, a negative consequence
is that abandoned, fractured or retained epicardial leads are considered unsafe for MRI, and these patients, for the rest of
their life, may no longer have the benefits afforded by MRI [9], [10]. 

Devices for use in MRI environments undergo testing
and are classified as MRI safe, MRI conditional, or MRI unsafe. For an MRI safe designation, the device would likely be
nonconductive, non-electrical, non-magnetic and otherwise present no known hazards in the MRI environment. MRI
conditional devices generally fall outside the MRI safe specification but have an established safety record under well-
defined conditions. Finally, a device is considered MRI unsafe if it poses known hazards to the patient within the MRI
environment. Reasons for the MRI unsafe classification for temporary, retained epicardial leads include concerns about
radiofrequency (RF) tissue heating, particularly at the tip of the wire, magnetic interactions with the magnetic field leading
to a translational force or torque on the wire, and the possibility of the unintentional stimulation of the heart due to electrical
currents generated in the wire [11]. 


While all of these are valid concerns and have been well established for larger permanent
pacing leads, there is less evidence indicating substantial risk from the smaller temporary pacing leads. As a result, the issue
has been controversial, with apparently conflicting consensus statements published by various organizations. Most notably,
in their 2017 statement, the Heart Rhythm Society (HRS) indicated that the presence of nonconditional epicardial leads was
a contraindication for MRI, though both previous and more recent consensus statements have disagreed indicating negligible
risk [1]–[3].


To begin to address this issue recently we systematically measured RF
heating induced during MRI of a tissue simulating phantom containing
post-surgical epicardial pacing leads of various lengths following
guidelines provided by the ASTM F2182-19 standard document [12].


This was done at CUMC Bergan Mercy Hospital using the Siemens 3T
Magnetom Trio and the work was presented at the 2021 American
Association of Physicists in Medicine (AAPM) annual meeting and
formed the basis of a Medical Physics Master’s Thesis [6].

Briefly,
temporary bipolar and unipolar epicardial leads of various lengths and
orientations were exposed to clinically-relevant imaging protocols
(Specific energy absorption rate (SAR) of 2 W/kg) and the maximum
temperature rise was recorded (Fig 1). While the maximum RF induced
temperature rise at the tip of the wire was significant for relatively long
(19 cm) leads, heating was not hazardous for wires less than 13 cm in
length. Measurements made with different lead orientations relative the
static magnetic field indicated that orientation was also a significant
factor, with maximum heating occurring when the implant wire was
aligned with the static magnetic field (parallel to the patient couch).
While these results suggest that RF heating will likely not be a problem
for relatively short leads, the additional potential hazards involving
translational forces and torques, and electrical stimulation via the retained
epicardial pacing lead have yet to be evaluated. Here we propose
experiments that will address the concerns related to translational forces
and torques, and in the future, we plan to continue the study to
characterize the potential for electrical stimulation.


Clinical MRI systems consist of the main magnet, usually a cylindrical
superconducting magnet, with field strengths typically of 1.5 or 3.0 T, 30 –
60 thousand times the Earth’s magnetic field [13]. 

This field is greatest and
most uniform at the center of the bore of the magnet, the imaging isocenter.
Outside of the magnet, the magnetic field strength diminishes by a factor of
1000 within a couple of meters from the magnet’s isocenter (Fig. 2).
Materials of concern could be either paramagnetic, which can be temporarily
magnetized, or ferromagnetic (permanent magnets). In either case, the
material will experience a force pulling in the direction of the spatial field
gradient and would move toward the imaging isocenter unless restrained.
The translational force on an object in an MRI scanner depends on the material’s
magnetic susceptibility ($\chi$, the degree to which an object can be magnetized),
the static (unchanging in time) magnetic field strength (B0) and rate of increase
in field strength with distance (i.e. the spatial gradient) of the field strength
($\nabla B0$) [13]–[15]. 

The maximal translational force is usually written as a
fraction of the objects weight as shown in Eq. 1, where $\rho$ is the objects
density and $\mu_0$, the magnetic permeability of free space, is a fundamental
constant [15].

While a precise force assessment will require in situ field measurements, we
can roughly estimate the forces at a location where the magnetic field and the
spatial gradient of the field is nearly maximum. Fig. 3 shows an estimate of
the magnetic field and field gradient for a 1.5 T system, clearly illustrating
that while the magnetic field (and torque) will be maximal at the imaging
isocenter, the field gradient and translational force will be maximal at the
entrance of the bore [14], [18]. 


For a nonmagnetic

stainless-steel wire ($\chi$~103 ppm, $\rho$ ~ 8000 kg/m3), the translational force is
expected to be ~ 30\% of its weight, a similar titanium wire would experience
a force of only 5\% of its weight, and a ferromagnetic material like pure nickel
would experience a force of nearly 20,000 times its weight. To appreciate the
significance of these values, an object experiencing a translational force equal
to its weight would accelerate to a speed of about 22 mph in 1 second and
become a projectile unless counteracted by an equivalent force - in this case
provided by the tissue that the lead is embedded in. 

According to the ASTM standard for magnetically
induced displacement forces in MRI, “for a device to be safe in the MR environment, the magnetically induced deflection
force and torque should be less than forces and torques to which the device may be safely exposed if it were not in a magnetic
field” [15]. 

Therefore, a device is generally accepted provided the translational and torque forces do not exceed the objects
weight.

1]J. H. Indik et al., “2017 HRS expert consensus statement on magnetic resonance imaging and radiation exposure
in patients with cardiovascular implantable electronic devices,” Hear. Rhythm, vol. 14, no. 7, pp. e97–e153, 2017.
2]P. Jabehdar Maralani et al., “MRI safety and devices: An update and expert consensus,” J. Magn. Reson. Imaging,
vol. 51, no. 3, 2020.
3]G. N. Levine et al., “Safety of magnetic resonance imaging in patients with cardiovascular devices: An American
heart association scientific statement from the committee on diagnostic and interventional cardiac
catheterization, council on clinical cardiology, and the council o,” Circulation, vol. 116, no. 24, pp. 2878–2891,
2007.
4]G. G. Hartnell, L. Spence, L. A. Hughes, M. C. Cohen, R. Saouaf, and B. Buff, “Safety of MR imaging in patients who
have retained metallic materials after cardiac surgery,” Am. J. Roentgenol., vol. 168, no. 5, 1997.
6]O. Aboyewa, “Measurement of Radiofrequency Induced Heating in Patients with Retained Post-Surgical
Epicardial Leads During MRI,” Creighton University, 2021.
7]M. M. Benjamin and C. A. Sorkness, “Practical and ethical considerations in the management of pacemaker and
implantable cardiac defibrillator devices in terminally ill patients,” Baylor Univ. Med. Cent. Proc., vol. 30, no. 2,
pp. 157–160, 2017.
8]M. C. Reade, “Temporary epicardial pacing after cardiac surgery: a practical review,” Anaesthesia, vol. 62, no. 4.
pp. 364–373, 2007.
9]E. Kanal, “Pacemakers in MRI for the neuroradiologist: Revisited,” Am. J. Neuroradiol., vol. 39, no. 5, pp. E54–E55,
2018.
10]R. G. Muthalaly, N. Nerlekar, Y. Ge, R. Y. Kwong, and A. Nasis, “MRI in patients with cardiac implantable electronic
devices,” Radiology, vol. 289, no. 2. Radiological Society of North America Inc., pp. 281–292, 2018.
11]P. G. Poh, C. Liew, C. Yeo, L. R. Chong, A. Tan, and A. Poh, “Cardiovascular implantable electronic devices: a
review of the dangers and difficulties in MR scanning and attempts to improve safety,” Insights Imaging, vol. 8,
no. 4, pp. 405–418, 2017.
12]ASTM 2182-19, “Standard Test Method for Measurement of Radio Frequency Induced Heating On or Near
Passive Implants During Magnetic Resonance,” Annu. B. ASTM Stand., vol. i, no. December, pp. 1–12, 2019.
13]J. M. B. Jerrold T. Bushberg, J. Anthony Seibert, Edwin M. Leidholdt, The Essential Physics of Medical Imaging, 3rd
ed. Philadelphia, PA: Lippincott Williams and Wilkins, 2011.
14]L. P. Panych and B. Madore, “The physics of MRI safety,” Journal of Magnetic Resonance Imaging, vol. 47, no. 1.
2018.
15]ASTM 2052-15, “Standard Test Method for Measurement of Magnetically Induced Displacement Force on
Medical Devices in the Magnetic Resonance Environment,” Annu. B. ASTM Stand., vol. 03, 2015.
16]A. Heinrich, J. Dorschel, M. Mohammad Mashoor, F. Guttler, and U. Teichgraber, “Development of an Apparatus
for Digital Measurement of Magnetically Induced Torque on Medical Implants to Facilitate the Application of the
ASTM F2213 Standard,” IEEE Trans. Biomed. Eng., vol. 66, no. 12, 2019.
17]A. 2213-17, “Standard Test Method for Measurement of Magnetically Induced Torque on Medical Devices in the
Magnetic Resonance Environment 1,” Annu. B. ASTM Stand., vol. 06, pp. 1–8, 2017.
18]E. Kanal, “Standardized Approaches to MR Safety Assessment of Patients with Implanted Devices,” Magn. Reson.
Imaging Clin. N. Am., vol. 28, no. 4, pp. 537–548, 2020.

%----------------------------------------------------------
%From Brights thesis (student before me)
%----------------------------------------------------------
Nearly 3.8 million people in the United State presently live with cardiac implantable
electronic devices (CIEDs) such as pacemakers and implantable cardioverter - defib-
rillators (ICD) [1].

Patients with permanent pacemakers (PPMs) may benefit greatly from magnetic res-
onance imaging (MRI) due to its advantage over other diagnostic modalities. Owing
to its excellent soft-tissue contrast, ability to image deep within the body, and the
lack of ionizing radiation [3, 4],

about 50\% to 75\% of patient
with CIEDs are estimated to require an MRI during their lifetime [6, 7].

three different types of magnetic fields including
the static, gradient and radiofrequency (RF) magnetic fields are required.

There are possibilities for mechanical displace-
ment of the pulse generator or dislodgement of the lead within the myocardium, re-
set/reprogramming of device, interference with pacing mode leading to asynchronous
pacing, changes in electrocardiogram and induced lead heating [2]. (main focus of ours in the mechanical displacements, first tranlational forces then torques)

At present,
no PPMs or ICDs have been declared MRI safe by the Food and Drug Administration
(FDA) [7]. Check


temporary epicardial leads placed during cardiac surgery or prior to
permanent implants are usually cut short at the skin, while a tiny portion of the
lead remains in the patient [12].

Due to safety concerns, CIEDs with presence of
abandoned, fractured or retained epicardial leads are classified as contraindications
for MRI examinations [10].

Clinical MRI systems consist of the main magnet, usually a cylindrical superconduct-
ing magnet, with field strength greater than 1 T [46].

A typical 1.5 T MRI scanner
commonly used for whole-body studies generates a magnetic field strength approxi-
mately 30,000 times greater than that of the earth’s magnetic field at its isocenter. B0 field strength rapidly diminishes as the dis-
tance from the isocenter increases.

At reasonable distances, the fringe field falls off
according to a dipole approximation at approximately 1/r3, approaching the strength
of the earth’s magnetic field ($\sim$0.05 mT) at a large distance [52]. (the 5 Gauss line)

materials can be
classified as diamagnetic, paramagnetic, and ferromagnetic based on their magnetic
susceptibility ($\chi$), Most medical devices are made of paramagnetic materials that
have small and positive magnetic susceptibility($\chi$ > 0) [18].
when in the
vicinity of the static
B0 field, they experience a translational force that attracts them
towards the center of the scanner or a torque that tends to rotate and cause them to
align with the field [18].

The action of the translation force (Ftrans) is due to the spatial variation of the
B0
field (measured in T/m).

B0 is usually uniform over the bore of the
magnet which implies that the displacement force is experienced before entry into the
magnet bore. In safety analysis, the significance of the translational force is usually
compared with gravitational pull (Fg) on the same object,
$$
Ftrans
=
Fg
\chi B0
(\mu_0g)\rho|\nabla B0|, (2.1)
$$
where $µ0 = 4π×10−7H/m$ is the permeability of free space, g = 9.8 m/s2 is the
acceleration due to gravity, $\rho$is the material density in kg/m3. Assuming a maximum
spatial gradient of about 10 T/m, according to equation 2.1, an object made of iron
in the vicinity of the main magnet at maximum saturation, i.e., $\chi$
B0 $\sim$2.2 T, would
experience a magnetic force about 250 times its weight and become a projectile [18].

1]Mina M Benjamin and Christine A Sorkness. “Practical and Ethical Consider-
ations in the Management of Pacemaker and Implantable Cardiac Defibrillator
Devices in Terminally Ill Patients”. In: Baylor University Medical Center Pro-
ceedings 30.2 (2017), pp. 157–160. doi: 10.1080/08998280.2017.11929566.
2]A W Korutz, A Obajuluwa, M S Lester, E N McComb, T A Hijaz, J D Collins,
S Dandamudi, B P Knight, and A J Nemeth. “Pacemakers in MRI for the Neu-
roradiologist”. In: American Journal of Neuroradiology 38.12 (2017), pp. 2222–
2230. doi: 10.3174/ajnr.A5314.
3]Volkan Acikel and Ergin Atalar. “Modeling of radio-frequency induced currents
on lead wires during MR imaging using a modified transmission line method”.
In: Medical Physics 38.12 (2011), pp. 6623–6632. doi: 10.1118/1.3662865.
4]Gregory H Griffin, Kevan J T Anderson, Haydar Celik, and Graham A Wright.
“Safely assessing radiofrequency heating potential of conductive devices using
image-based current measurements”. In: Magnetic Resonance in Medicine 73.1
(2015), pp. 427–441. doi: 10.1002/mrm.25103.
6]Ron Kalin and Marshall S Stanton. “Current clinical issues for MRI scanning of
pacemaker and defibrillator patients”. In: Pacing and clinical electrophysiology
28.4 (2005), pp. 326–328.
7]Rahul G Muthalaly, Nitesh Nerlekar, Yin Ge, Raymond Y Kwong, and Arthur
Nasis. “MRI in Patients with Cardiac Implantable Electronic Devices”. In:
Radiology 289.2 (2018), pp. 281–292. doi: 10.1148/radiol.2018180285.
10]Julia H Indik, J Rod Gimbel, Haruhiko Abe, Ricardo Alkmim-Teixeira, Ulrika
Birgersdotter-Green, Geoffrey D Clarke, Timm-Michael L Dickfeld, Jerry W
Froelich, Jonathan Grant, David L Hayes, Hein Heidbuchel, Salim F Idriss,
Emanuel Kanal, Rachel Lampert, Christian E Machado, John M Mandrola,
Saman Nazarian, Kristen K Patton, Marc A Rozner, Robert J Russo, Win-
Kuang Shen, Jerold S Shinbane, Wee Siong Teo, William Uribe, Atul Verma,
Bruce L Wilkoff, and Pamela K Woodard. “2017 HRS expert consensus state-
ment on magnetic resonance imaging and radiation exposure in patients with
cardiovascular implantable electronic devices”. In: Heart Rhythm 14.7 (July
2017), e97–e153. doi: 10.1016/j.hrthm.2017.04.025.
12]G G Hartnell, L Spence, L A Hughes, M C Cohen, R Saouaf, and B Buff.
“Safety of MR imaging in patients who have retained metallic materials after
cardiac surgery.” In: American Journal of Roentgenology 168.5 (May 1997),
pp. 1157–1159. doi: 10.2214/ajr.168.5.9129404.
18]Lawrence P Panych and Bruno Madore. “The physics of MRI safety”. In:
Journal of Magnetic Resonance Imaging 47.1 (2018), pp. 28–43. doi: 10 .
1002/jmri.25761.
46]Jerrold T Bushberg and John M Boone. The essential physics of medical imag-
ing. Lippincott Williams \& Wilkins, 2011, pp. 438–444, 495–499.
52]P L Carson, S R Thomas, M Koskinen, M Lassen, W Pavlicek, R R Price, and
M J Bronskill. “Site Planning for Magnetic Resonance Imaging”. In: NMR in
Medicine, AAPM Technical Monograph, American Association of Physicists
in Medicine, Thomas SR, ed., New York, NY (1985), pp. 1902–1907.