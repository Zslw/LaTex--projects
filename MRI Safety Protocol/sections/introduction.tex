Magnetic resonance imaging (MRI) is an essential diagnostic tool, it provides superior soft tissue contrast without the safety hazzards that ionizing radiation introduces. Like every other medical imaging modality, safety is a main concern for its use in a clinical setting. In the case of MRI, the powerful static magnetic field and associated spatial gradients impose strict safety constraints on patients with implanted medical devices. Currently more than 4.5 million individuals worldwide live with cardiac implantable electronic devices (CIEDs), including pacemakers and implantable cardioverter-defibrillators (ICDs)  \citep{aboyewa2021,benjamin2017}. 

It is in our interest to study a particular subset of CIEDs, temporary epicardial pacing leads, around the MRI environment. These provisional leads are frequently placed during open-heart surgery to regulate cardiac rhythm postoperatively, then are typically cut at the skin surface leaving behind a short length of wire embedded in the tissue \citep{reade2007}. In general, a device labeled as MRI safe is expected to be nonconductive, non-electrical, non-magnetic, and to present no known hazards in the MRI environment. However, this classification does not apply to most epicardial leads, as they are rarely physically characterized after removal. By the time they are cut, these leads have fulfilled their clinical purpose, and systematic safety evaluations are seldom performed. Although retained leads are nonfunctional and relatively short, they are frequently classified as MRI unsafe due to concerns over radiofrequency (RF) heating, magnetically induced translational forces and torques, or inadvertent cardiac stimulation \citep{poh2017,muthalaly2018}.

The concern that has drawn the most attention among these is RF-induced heating.  Previous work conducted at Creighton University evaluated the heating effects of retained leads in a tissue-simulating phantom under clinical imaging conditions. This study found that short leads (<13 cm) posed no significant thermal hazard during MRI at a specific absorption rate (SAR) of 2 W/kg and is throughtfully described as Master Thesis in 2021 \citep{haddixProposal,astmF2182,aboyewa2021}.

The next step is to asses magnetically induced translational forces. These forces are described as a function of the static field strength ($B_0$), its spatial gradient ($\nabla B_0$), and the magnetic susceptibility ($\chi$) and density ($\rho$) of the material. A well known and expected behavior is the magnetic pull on high susceptibility materials toward regions of higher field strength, like near the bore entrance of the scanner where the spatial gradient is highest \citep{aboyewa2021,bushberg2011,panych2018}. For instance, a nonmagnetic stainless steel wire with $\chi \approx 103$ ppm may experience a force equivalent to $\sim$30\% of its weight, whereas ferromagnetic materials like pure nickel can experience forces exceeding 20,000 times their own weight. That is more than enough to become dangerous projectiles unless restrained \citep{aboyewa2021,panych2018}.

ASTM F2052-21 offers a standardized technique for assessing magnetically induced displacement force on medical equipment in order to set safety standards for such situations. This standard states that a device is MR Conditional if the deflection force it encounters in the MRI environment is less than the force of gravity upon it. This is usually verified if the deflection angle from vertical is less than 45° \citep{stoianovici2024,astmF2052}.

Despite these established protocols, many retained leads have not undergone formal mechanical testing due to their temporary nature and variability in material and geometry.  The primary aim of this work is to show and validate a custom-built, lab-scale, MR-compatible translational force testing platform and demonstrate its capability to produce quantitative, ASTM-aligned safety data. The device allows to precisely place a reference rod or pacing wires at clinically significant points in the MRI fringe field where translational forces are at their highest. 

Measurements are interpreted in the context of magnetic susceptibility theory and compared to relevant thresholds from the literature \citep{haddixProposal,astmF2052}. These techniques will help with more thorough evaluations of MRI safety in patients with retained temporary epicardial leads and set the stage for a future clinical adaptation of the protocol employing a 1.5 T MRI scanner. This method aids in the improvement of device classification and the creation of uniform experimental procedures for the assessment of torque and translational safety.