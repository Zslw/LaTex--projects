\section{I. Laboratory Protocol (Physicist Focus)}
\subsection*{Materials and Setup}
\begin{itemize}
	\item \textbf{PVC Base:} Rectangular prism with leveling screws.
	\item \textbf{Pendulum Arm:} 1.5 m vertical tube, light string suspension.
	\item \textbf{Lead Mount:} PVC frame replacing magnetic rod, with fixed hangers.
	\item \textbf{Ruler:} Positioned behind pendulum to measure z-displacement.
	\item \textbf{Solenoid:} Coil capable of up to 5 A current. Represents MRI B$_0$ gradient.
	\item \textbf{Optional Guide Rails:} PVC slots to limit pendulum rotation and allow angle testing.
\end{itemize}

\subsection*{Procedure}
\begin{enumerate}
	\item Level base with screws; confirm vertical alignment of the tube.
	\item Attach the lead to the frame and ensure central positioning.
	\item At each test position (distance from solenoid), record resting position.
	\item Energize solenoid and record max z-displacement.
	\item Compute angle: $\theta = \arcsin(\Delta z / L)$
	\item Compute force: $F = mg \tan(\theta)$
	\item Repeat 3--5 times per position; calculate mean, standard deviation.
\end{enumerate}

\subsection*{Uncertainty Notes}
\begin{itemize}
	\item Waiting for swing stabilization increases precision, but slows throughput.
	\item Trade-off documented: rapid swings accepted, wider uncertainty bounds.
	\item High current causes sudden attraction $\rightarrow$ mitigate by spacing solenoid.
\end{itemize}

\section{II. Clinical Translation}
\begin{itemize}
	\item In MRI, B$_0$ is always on; increasing solenoid current models proximity to bore.
	\item Couch marks every 10--15 cm simulate test spacing.
	\item Rotational setup mirrors patient-specific orientations (axial, sagittal, oblique).
\end{itemize}

\section{III. Radiologist Collaboration}
\subsection*{MRI Clearance Flowchart (ASCII-safe)}
\begin{verbatim}
	[Patient has epicardial lead]
	|
	[Is lead model known?]
	|
	Yes   No
	|     |
	[Check]   [Refer to Physicist]
	|
	[Force < 0.01 N?]
	|
	Yes   No
	|     |
	[MRI OK] [MRI Contraindicated]
\end{verbatim}

\subsection*{Workflow Proposal}
\begin{itemize}
	\item Maintain internal registry with lead ID, orientation, max force, test conditions.
	\item Share summaries as PDFs; link to PACS or EPIC where feasible.
	\item Annual Radiology--Physics review to update protocols and registry.
\end{itemize}

\section*{Appendices (To Be Added)}
\begin{itemize}
	\item \textbf{Appendix A: Mechanical Setup} -- diagrams, photos, 3D model links.
	\item \textbf{Appendix B: Calibration Logs} -- pendulum swing logs, trigonometry, force estimates.
	\item \textbf{Appendix C: Clinical Registry Template} -- lead name, test ID, orientation, force, MRI recommendation.
\end{itemize}