EVALUATION PLAN
Device Fabrication and Protocol Development. After we have fabricated the devices described in Aims 1 and 2 and
established their utility, we will prepare publications providing appropriate schematics and materials as well as protocols
for their effective use. This will also be disseminated through presentations at local, regional, and national conferences.

Data Analysis. The successful completion of the project requires establishing the measurement accuracy for the static B0
field, the field gradient, translational forces, and torques (by two techniques). For each of 10 Z-axis locations, approximately
60 measurements of B0 will be made and repeated on at least three separate occasions to verify consistency. Each of the
translational force and torque measurements will be repeated approximately 3-5 times at each location within the MRI
scanner tested (up to 10 axial measurement sites with 5 positions within the plane (left, middle, right, top, bottom). The
mean and the standard error will be reported and results for each epicardial lead will be compared to the established safety
standards.

Expected Outcomes, Benchmarks for Success, and Projecting Future Directions. By completing Aims 1 and 2, we will
develop the apparatus and methodology to 1) characterize the spatial distribution of the magnetic field in a clinical MRI
scanner, 2) measure the maximum translational forces and torques on a representative subset of temporary, post-surgical
epicardial pacing leadings, and 3) assess the safety of the leads regarding static B0 magnetic field interactions according to
established standards. These results will be interpreted along with our prior measurements of potential hazards of RF heating
to provide a more complete assessment. The techniques and the collaboration strengthened through this study will also pave
the way for a future study of the potential for electrical stimulation by retained, post-surgical epicardial pacing leads during
MRI. We plan to publish these results within respected medical physics and radiology journals, and will disseminate the
results at national meetings, including the American Association of Physicists in Medicine (AAPM) annual meeting.
Progress made through this work will enable applications for future funding such as NIH R15 and R21 proposals through
NIGMS and NIBIB.
