\subsection{designs}
Using designs compatible with ASTM protocols, we will fabricate two devices that will enable accurate
measurements of the magnetic field, field gradient, and translational forces within an MRI scanner. First, we will design
and build an MR compatible platform that can be easily positioned on the patient couch of an MRI scanner. The platform
will provide a well-defined spatial array of slots to hold a gaussmeter at known locations relative to the axis of the scanner
for magnetic field strength (B0) and spatial gradient ($\nabla B0$) measurements. 

(DONE) explain the prototype instead add some figures and photos of it, 

Next, we will design and fabricate an MR
compatible testing device that will be used to measure the translational force on a variety of pacing leads (pediatric, adult,
unipolar, and bipolar of lengths varying from 6.3 cm to 19 cm) held at a desired location within the scanner. Working plans
are available within ASTM documents and published reports and the basic design discussed below (Methods) can readily
be fabricated in Creighton’s machine shop [15], [19]–[21]. Measurements made with these devices will allow direct
assessment of the translational force on the epicardial leads (and a Titanium rod, used as a standard reference), which will
then be compared with theoretical prediction described in Eq. 1

(Main focus of this report) I used this device, i got some schematics and steps to work with it

Aim 2. Using designs compatible with ASTM protocols, we will fabricate a device that will enable accurate measurements
of the torque on a wire segment placed in the static magnetic field of an MR system. The device will enable two types of
torque measurements. The first approach will implement a relatively simple, low-friction surface testing method (Methods).
While this is a quick and straightforward technique, ideal for clinical use, it will only establish an upper limit of the maximal
torque on the epicardial pacing lead. Therefore, a second more quantitative approach will also be used to measure the torque
on the same epicardial pacing leads and Ti Rod studied in Aim 1.

(Contraption exists, dont know how it works)


\subsection{Methods}

Measurement of Magnetic Field and Field Gradient. A commercial
gaussmeter (Gaussmeter model GM2, AlphaLab, Inc., Salt Lake City,
UT) will be used to measure the magnetic field at fixed locations within
the MRI scanner. When the probe of the gaussmeter is placed in a
magnetic field, a voltage is induced between its parallel faces (Hall
effect) which is directly proportional to the magnetic flux at that
position. To ensure the spatial precision of these measurements, we will
fabricate a rigid, PVC fixture containing a base plate and a vertically
standing circular plate, similar to that shown in Fig. 4. The vertical circular plate will
have an array of slots with known X and Y positions to accept the gaussmeter probe for
spatial magnetic field measurements. We will repeat the measurements by translating the
fixture through various Z-locations in the bore of the MRI system (using axial patient
couch shifts at 10 cm increments starting from the entrance of the bore). The measured
magnetic fields at various spatial positions can be further used to estimate and map the
spatial field gradient using a finite-difference estimate. A similar method has been used
to map the magnetic field and magnetic field gradient for the MRI in another study [22].

(done) explain how to use it in the clinic

Measurement of Magnetic Field and Field Gradient. A commercial
gaussmeter (Gaussmeter model GM2, AlphaLab, Inc., Salt Lake City,
UT) will be used to measure the magnetic field at fixed locations within
the MRI scanner. When the probe of the gaussmeter is placed in a
magnetic field, a voltage is induced between its parallel faces (Hall
effect) which is directly proportional to the magnetic flux at that
position. To ensure the spatial precision of these measurements, we will
fabricate a rigid, PVC fixture containing a base plate and a vertically
standing circular plate, similar to that shown in Fig. 4. The vertical circular plate will
have an array of slots with known X and Y positions to accept the gaussmeter probe for
spatial magnetic field measurements. We will repeat the measurements by translating the
fixture through various Z-locations in the bore of the MRI system (using axial patient
couch shifts at 10 cm increments starting from the entrance of the bore). The measured
magnetic fields at various spatial positions can be further used to estimate and map the
spatial field gradient using a finite-difference estimate. A similar method has been used
to map the magnetic field and magnetic field gradient for the MRI in another study [22].

Measurement of Translational Forces . The measurement of the translational force
exerted on pacing leads during MRI will be done in accordance with the ASTM protocol
[15]. Briefly, The test specimen is suspended by a light string in presence of the magnetic
field. The force exerted by the magnetic field (Fm) acts along the axis of the bore (in the
direction of the field gradient). The force due to its own weight (mg) acts vertically
downwards, Ts is the tension on the string and $\alpha$ is the angle made by the string with the
horizontal (Fig. 5). Under equilibrium conditions, the translational force on the test
specimen due to the MRI system will be given by Eq 8.Thus, the ratio of translational
force to the sample weight is the tangent of the angle made by the string with the horizontal (i.e. the deflection angle).
Therefore, we will measure the deflection angle experimentally and calculate the translational force using Eq. 8. Because
these leads range in length from 6 – 19 cm and are very flexible, a lightweight form will be developed to hold the lead and
allow attachment to the string. The total weight will be used to calculate the magnetic force. To verify the accuracy of the measurement, measurements will be made with the lead in both linear, (extended) and tight coil configurations. Results
from both configurations will be compared with the theoretical predictions of Eq. 1. Measurements will be made at 10
cm increments along the MRI bore axis to ensure the location of the maximum deflection is measured accurately
and increase measurement sensitivity. The test fixture will be positioned such that the center of mass of the test
device is at the test location with known magnetic field and gradient field. Since the magnetic field strength and
the gradient field will be acting along z-axis (other components of gradient magnetic field are expected to be
negligible compared to that along z-axis), the magnetically induced translational force will also act along z-
direction.
Measurement of the deflection angle. To measure the
deflection angle $\alpha$ (Eq. 8), a PVC test fixture will be
fabricated that is capable of holding the test device in
the proper position during the measurement. This will
be similar to that shown in Fig 6. The test fixture will be
placed on a base plate with an adjustable vertical post.
All coordinates will be determined by MRI couch shifts
referenced from the MRI isocenter. An inflexible string
hanging from the post will hold the test device. A
protractor with 1 deg graduated markings will be mounted
to the vertical post for reading out the deflection angles.
The test device will be suspended such that it coincides
with the 0 deg marking on the protractor. When placed in

the test location, the test device will experience the
translational force and hence will deflect. The deflection angle $\alpha$ from the vertical direction to nearest 1 deg will be recorded.
For each test device and the test location, this measurement will be repeated at least three times. The average deflection
angle from the trials will be calculated and used to determine the ratio of translational force experienced by the test device
to the weight of the test device. If the average value of deflection angle is less than 45deg, then the translational force is less
than its own weight; hence is not a significant clinical risk. However, if the deflection angle exceeds 45deg, then the
translational force on the lead can present a significant risk to the patient.


15]ASTM 2052-15, “Standard Test Method for Measurement of Magnetically Induced Displacement Force on
Medical Devices in the Magnetic Resonance Environment,” Annu. B. ASTM Stand., vol. 03, 2015.

19]F. G. Shellock, J. A. Tkach, P. M. Ruggieri, and T. J. Masaryk, “Cardiac pacemakers, ICDs, and loop recorder:
Evaluation of translational attraction using conventional (‘long-bore’) and ‘short-bore’ 1.5- and 3.0-Tesla MR
systems,” J. Cardiovasc. Magn. Reson., vol. 5, no. 2, pp. 387–397, 2003.
21]F. G. Shellock, E. Kanal, and T. B. Gilk, “Regarding the value reported for the term ‘spatial gradient magnetic field’
and how this information is applied to labeling of medical implants and devices,” Am. J. Roentgenol., vol. 196, no.
1, pp. 142–145, 2011.
22]A. Ferreira, “Predicting Angular Displacement of Medical Devices in a MRI Scanner Bore Using COMSOL
Multiphysics.,” Drexel University, 2017.