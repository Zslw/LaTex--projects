\subsection{Current Limitations}

FDG-PET/CT is widely used, but it has limitations. The major one is resolution, this makes detecting micrometastases realky difficult. Another limitation is in the tracer it self, FDG lack the specificity in distinguishing inflammatory from malignant lesions and can lead to diagnostic uncertainty. Furthermore, accessibility remains an issue, as the high costs associated with FDG-PET/CT and its limited availability in certain regions hinder its adoption in resource-limited settings.

\subsection{Future Advancements}

On the side of chemistry and nuclear physics, the development of radiotracers, such as 68Ga-FAPI and [18F]FMISO, offers solutions to FDG’s specificity challenges. These tracers target unique tumor characteristics, such as hypoxia and fibroblast activity \cite{Deng2021}. Then on the resolution problem, technologies like PET/MR provide enhanced soft-tissue contrast and reduced radiation exposure, making them promising alternatives for pancreatic cancer imaging. Additionally, Total Body PET/CT systems promise comprehensive whole-body imaging in a single scan, improving detection of micrometastases and overall efficiency.
