Pancreatic cancer is a type of cancer that starts to spread out form the pancreas. The pancreas is situated behind the stomach and its functions include secreting of digestion enzymes and regulating blood sugar. What makes pancreatic cancer really serious is its \textit{poor prognosis} this means that compared to other cancers this one is harder to treat successfully, harder to detect or prevent from growing.

This type of cancer is one of the most challenging in the field of oncology, it has a notably high mortality rate and low survival rates. In the US, pancreatic cancer is currently the 10th most commonly diagnosed cancer, with an estimated 66,440 new cases anticipated in 2024, accounting for 3.3\% of all new cancer cases \cite{SEER2024}. The incidence rate is about 13.5 per 100,000 men and women, with annual increases of about 1\% since the early 2000s. And sadly, it is responsible for 8.5\% of all cancer-related deaths, with an estimated 51,750 deaths expected in 2024 \cite{SEER2024}.

This similar scenario is observed across the world, for example in Mexico it ranks 12th in incidence, and 7th in mortality \cite{GlobocanMexico}. In China incidence ranks 8th. Overall, Pancreatic cancer is the fifth most common cause of cancer-related deaths in South Korea, and the fourth leading cause of cancer-related deaths in the US and Europe. \cite{NCCNGuidelines}

Moreover, pancreatic ductal adenocarcinoma (PDAC), one of the most lethal human cancers that conforms 85\% of pancreatic cancers, is estimated to be the second leading cause of cancer-related deaths by 2030. \cite{Li2022,Cancers2023}.

This is a general concern for health professionals and it reflects the limited or "slow" advances in diagnosis and treatment for this malignancy.

%\subsection{Challenges in Early Detection}

Given this statistics, it is crucial for us to quickly identify and treat this type of cancer. This can be quite challenging as it has an aggressive nature and asymptomatic progression. Usually this malignancy doesn't manifest and this leads to diagnoses in more advanced stages and by then surgery is rarely viable \cite{Pubmed30721664}.

There specific reasons why PDAC early detection is challeging has its root in different factors. On the clinical side of things, the initial stages of pancreatic cancer often go unnoticed as symptoms are minimal and non-specific. This early stage end quickly, the rapid progression and agrassive nature of the disease contributes to the complexity, by this time symtoms are noticible but may be to late.   Additionally, the deep location in the anatomy of the pancreas complicates physical examination. 

On the technical part, unlike other malignancies, pancreatic cancer lacks a standardized and reliable screening protocol for the general population, reducing the likelihood of early intervention. Futhermore, existing imaging modalities often fail to detect smaller lesions or early-stage pancreatic cancer \cite{Pubmed30721664}.It is worth pointing out that even surveillance with available techniques is sometimes not recommended if the patient has not developed any symptoms.

Given these diagnostic limitations, there is an increasing demand for imaging solutions can appropriately stage the disease and detect it early in order to improve patient outcomes. Recent literature emphasizes the need for screening programs for asymptomatic, high-risk individuals with non-invasive precursor lesions, rather than those in advanced stages \cite{Cancers2023}.



%\subsection{The Role of Imaging in Pancreatic Cancer Management}

Now, how can we detect this malignancy given all this challenges? Medical imaging is particularly useful in this case, it is able to provide information for treatment planning and assessing resectability. Current guidelines, including those by the National Comprehensive Cancer Network (NCCN) and the European Society for Medical Oncology (ESMO), point out that computed tomography (CT) is the primary imaging modality for assessing pancreatic cancer \cite{NCCNGuidelines}. However, emerging modalities such as endoscopic ultrasound (EUS) and magnetic resonance imaging (MRI) are quickly being more common as supplementary techniques for early-stage evaluation. There is data that compares these modalities in terms of strength, weaknesses and performance that suggests that a hybrid approach like PET/CT will allow for the detection of increased glucose metabolism, offering insights beyond structural abnormalities. \cite{life13102044}

\subsection{FDG PET/CT}

A promising modality for cancer diagnosis and staging is Positron emission tomography/computed tomography (PET/CT). While PET scans are able to capture cellular metabolic changes, CT provides anatomical mapping, this hybrid approach of techniques enables a complete assessment of the disease. Recent studies explore the main advantages and procedures involved using PET/CT with 18-fluorodeoxyglucose ([18F]FDG) a tracer to asses pancreatic cancer. They highlight that the unique advantage of [18F]FDG-PET/CT is in its sensitivity to the metabolic activity of cancer cells, providing valuable insights into both the presence of tumors and their potential aggressiveness \cite{Pu2021}.

Because of the asymptomatic and aggressive nature of PDAC, giving a clear diagnostic is challenging even now with imaging modalities. The advancements in this modalities like [18F]FDG-PET/CT, are very promising improving early detection rates and treatment outcomes. This paper will explore the current applications of [18F]FDG-PET/CT in pancreatic cancer diagnosis and staging, with a focus on overcoming diagnostic limitations and enhancing the accuracy of early detection.
