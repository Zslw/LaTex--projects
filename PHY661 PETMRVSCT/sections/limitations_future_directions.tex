% ============================
% Section: Limitations and Future Directions
% ============================


%\subsection{Challenges of PET/MR Integration}

Despite its advantages, PET/MR has logistical and technical difficulties. The most obvious ones are the extended acquisition times and higher costs. Second to that, consider the limited availability in clinical settings and integration with current techniques. Furthermore, the lack of standardized attenuation correction methods and sensitivity to motion artifacts is a huge deterrent. Addressing these issues will require advancements in imaging technology, standarized protocols and better workflow integration.\cite{pichler2008,Prakken2023}

%\subsection{Need for AAPM Guidelines}
To fully realize the potential of PET/MR in radiotherapy, comprehensive guidelines from professional organizations like AAPM are necessary. These guidelines (like they do for PET/CT) should address dosimetry protocols, motion management techniques, and the development of pseudo-CT methods for attenuation correction or alternatives. Further research is needed to establish the clinical efficacy of PET/MR in routine practice and optimize its role in radiotherapy workflows.
