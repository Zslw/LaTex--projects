% ============================
% Section: Discussion
% ============================


%\subsection{Improved Targeting Accuracy}
Better tumor delineation can be archived with the use of hybrid imaging modalities. They provide unique features like the integration of anatomical and biological data that allows for improvements. More precise treatment planning is possible as this approach advances; this is especially true for body parts with intricate tumor borders or areas that move a lot, like the liver.



%\subsection{Implications for Liver Cancer Management}
By considering the potential of new and linked technologies on liver cancer management will transform therapeutic approaches, soft tissue contrast and functional imaging of PET/MR is really promising. cases with small, poorly defined and unresectable tumors will be benefited from these advantages and developments. As PET/MR becomes more widely adopted, its role in adaptive radiotherapy workflows and personalized treatment regimens could improve survival rates and quality of life for liver cancer patients.

The potential of PET/CT and PET/MR imaging in radiotherapy planning is further illustrated through a case study on $^{90}\text{Y}$ microsphere therapy, detailed in Appendix \ref{sec:case}. This case highlights key differences in dosimetry, motion management, and imaging parameters between PET/CT and PET/MR, underscoring their respective advantages in radiotherapy workflows.