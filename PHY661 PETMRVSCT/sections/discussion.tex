% ============================
% Section: Discussion
% ============================


%\subsection{Improved Targeting Accuracy}
Better tumor delineation can be archived with the use of hybrid imaging modalities. They provide unique features like the integration of anatomical and biological data that allows for improvements. More precise treatment planning is possible as this approach advances; this is especially true for body parts with intricate tumor borders or areas that move a lot, like the liver.

By capturing metabolic and vascular behavior, the combination of functional imaging modalities including Diffusion-Weighted Imaging (DWI) and Dynamic Contrast-Enhanced (DCE) imaging allows for improved tumor characterisation in PET/MR.

%\subsection{Implications for Liver Cancer Management}
By considering the potential of new and linked technologies on liver cancer management will transform therapeutic approaches, soft tissue contrast and functional imaging of PET/MR is really promising. cases with small, poorly defined and unresectable tumors will be benefited from these advantages and developments. As PET/MR becomes more widely adopted, its role in adaptive radiotherapy workflows and personalized treatment regimens could improve survival rates and quality of life for liver cancer patients.

The idea of biological target volumes (BTV) is also introduced by the incorporation of PET/MR into radiation operations, opening the door to sophisticated techniques like dose painting. This approach allows higher doses to be delivered to treatment-resistant regions while sparing healthy tissue, ultimately improving therapeutic outcomes. Additionally, pseudo-CT creation in PET/MR can simplify processes, cutting down on treatment planning time and complexity.


%In Appendix \ref{sec:case}, a case study on $^{90}\text{Y}$ microsphere treatment is shown to further demonstrate the usefulness of PET/CT and PET/MR imaging in radiation planning. This case highlights key differences in dosimetry, motion management, and imaging parameters between PET/CT and PET/MR, underscoring their respective advantages in radiotherapy workflows.