% ============================
% Section: Conclusion
% ============================
This paper compares the advantages and limitations of PET/CT and PET/MR in liver cancer radiotherapy, it has a focus on their impact on tumor delineation, dosimetry, and treatment planning. PET/CT remains the gold standard for many clinical applications and PET/MR demonstrates significant potential in improving targeting accuracy and providing superior soft-tissue contrast and additional functional imaging capabilities, such as Diffusion-Weighted Imaging (DWI) and Dynamic Contrast-Enhanced (DCE) imaging.

Dose painting and other adaptive radiotherapy methods are made possible by the special potential for improved tumor characterisation that come with PET/MR integration. These advances are particularly relevant for complex cases involving small, poorly defined, or unresectable liver tumors. Despite these advantages, obstacles including longer acquisition times, more expenses, and the lack of standardized attenuation correction techniques prevent PET/MR from being widely used.


Future research and the development of standardized protocols will set the way in overcoming current limitations and ensuring these technologies achieve their full clinical potential in precision radiotherapy.